% A ConTeXt document

\startchapter[title=Documentation]

There are two ways to document a given \ConTeXt\ module, by either providing 
documentation embedded in the module definition file, or conversely by 
writing various \ConTeXt\ documents. 

\startsection[title=Internal documentation]

The \ConTeXt\ module documenation system makes use \TeX\ comments embedded 
inside the module definition files to provide literate programming style 
documentation of the macro definitions for any given module. 

There are four distinct types of document comments, in each case 

\startitemize

\item Standard \TeX\ comments begin with a \type{%} character and continue 
until the end of the current line of text. Any such comments will be embedded 
into the definitional section of the module documentation. For example:

\starttyping

% begin info
%
% title   : ConTests: Unit testing for \CONTEXT\
% comment : Unit testing for \CONTEXT\ using LunaTest
% status  : under development, mkiv only
%
% end info

\stoptyping

\item In a Copyright sections, each line begins with \type{%C} and continues 
to the end of the current line. For example:

\starttyping

%C Copyright (C) 2017 PerceptiSys Ltd (Stephen Gaito)
%C
%C Permission is hereby granted, free of charge, to any person obtaining
%C a copy of this software and associated documentation files (the 
%C "Software"), to deal in the Software without restriction, including 
%C without limitation the rights to use, copy, modify, merge, publish, 
%C distribute, sublicense, and/or sell copies of the Software, and to 
%C permit persons to whom the Software is furnished to do so, subject
%C to the following conditions:
%C
%C The above copyright notice and this permission notice shall be included 
%C in all copies or substantial portions of the Software.
%C
%C THE SOFTWARE IS PROVIDED "AS IS", WITHOUT WARRANTY OF ANY KIND, EXPRESS 
%C OR IMPLIED, INCLUDING BUT NOT LIMITED TO THE WARRANTIES OF 
%C MERCHANTABILITY, FITNESS FOR A PARTICULAR PURPOSE AND NONINFRINGEMENT. 
%C IN NO EVENT SHALL THE AUTHORS OR COPYRIGHT HOLDERS BE LIABLE FOR
%C ANY CLAIM, DAMAGES OR OTHER LIABILITY, WHETHER IN AN ACTION OF CONTRACT, 
%C TORT OR OTHERWISE, ARISING FROM, OUT OF OR IN CONNECTION WITH THE 
%C SOFTWARE OR THE USE OR OTHER DEALINGS IN THE SOFTWARE.

\stoptyping

\item In a 

\stopitemize

The following description of the documentation comments has been taken from 
the \type{mtx-modules.lua} file in the 
\type{<contextInsallation>/tex/texmf-context/scripts/context/lua} directory. 

\starttyping

-- Documentation can be woven into a source file. This script can generates
-- a file with the documentation and source fragments properly tagged. The
-- documentation is included as comment:
--
-- %D ......  some kind of documentation
-- %M ......  macros needed for documenation
-- %S B       begin skipping
-- %S E       end skipping
--
-- The generated file is structured as:
--
-- \starttypen
-- \startmodule[type=suffix]
-- \startdocumentation
-- \stopdocumentation
-- \startdefinition
-- \stopdefinition
-- \stopmodule
-- \stoptypen
--
-- Macro definitions specific to the documentation are not surrounded by
-- start-stop commands. The suffix specification can be overruled at runtime,
-- but defaults to the file extension. This specification can be used for language
-- depended verbatim typesetting.
--
-- In the mkiv variant we filter the \module settings so that we don't have
-- to mess with global document settings.

\stoptyping

\stopsection

\stopchapter
